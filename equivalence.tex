\section{Full Equivalence Proof}
\label{sec:equivalence}

\subsection{Preliminaries}
\label{sec:equivalence:preliminaries}

\subsubsection{I/O Automata}
\label{sec:equivalence:preliminaries:ioa}

We model each component in our system as an I/O automaton
(IOA)~\cite{lynch1987ioa,lynch1996da}, a type of state machine. Each
transition of an IOA is labeled with an \textit{action}, which can be an
\textit{input}, \textit{output}, or \textit{internal} action. Input and output
actions allow the automaton to interact with other IOA and the environment. We
assume input actions are not controlled by an IOA---they may arrive at any time.
Conversely, output and internal actions are \textit{locally controlled}---an IOA defines
when they can be performed.

To specify an IOA, we must first specify its \textit{signature}. A signature $S$ is
a tuple comprising three disjoint sets of actions: input actions $\actin(S)$,
output actions $\actout(S)$, and internal actions $\actint(S)$. We also define
$\localacts(S) = \actout(S) \cup \actint(S)$ as the set of locally controlled actions, $\extacts(S) = \actin(S) \cup \actout(S)$ as the set of external actions, and $\acts(S) = \actin(S) \cup \actout(S) \cup \actint(S)$ as the set of all actions.

Formally, an I/O automaton $A$ comprises four items:
\begin{enumerate}
\item a signature $\sig(A)$,
\item a (possibly infinite) set of states $\states(A)$,
\item a set of start states $\start(A) \subseteq \states(A)$, and
\item a transition relation $\trans(A) \subseteq \states(A) \times
  \acts(\sig(A)) \times \states(A)$.
\end{enumerate}
Since inputs may arrive at any time, we assume that for every state $s$ and
input action $\pi$, there is some $(s, \pi, s^\prime) \in \trans(A)$.

An \textit{execution} of an I/O automaton $A$ is a finite or infinite
sequence of alternating states and actions $s_0,\pi_1,s_1,\ldots$ such that for
each $i \geq 0$, $(s_i, \pi_{i+1}, s_{i+1}) \in \trans(A)$ and $s_0 \in \start(A)$. Finite executions always end with a state.

Given an execution $\alpha$, we can also define
its \textit{schedule} $\sched(\alpha)$, which is the subsequence of just
the actions in $\alpha$. Similarly, an execution's \textit{trace}
$\trace(\alpha)$ is the subsequence of just the external actions $\pi \in
\extacts(A)$.

\subsubsection{Composition and Projection}
\label{sec:equivalence:preliminaries:composition}

To compose
two IOA, they must be \textit{compatible}. Formally, a finite set of
signatures $\{S_i\}_{i \in I}$ is compatible if for all $i,j \in I$ such that $i \neq j$:
\begin{enumerate}
\item $\actint(S_i) \cap \acts(S_j) = \emptyset$, and
\item $\actout(S_i) \cap \actout(S_j) = \emptyset$.
\end{enumerate}
A finite set of automata are compatible if their signatures are compatible.

Given a set of compatible signatures, we define their \textit{composition} $S =
\prod_{i \in I} S_i$ as the signature with $\actin(S) = \bigcup_{i \in I}
\actin(S_i) - \bigcup_{i \in I} \actout(S_i)$, $\actout(S) = \bigcup_{i \in I}
\actout(S_i)$, and $\actint(S) = \bigcup_{i \in I} \actint(S_i)$.

The composition of a set of compatible IOA yields the automaton $A =
\prod_{i \in I} A_i$ defined as follows:
\begin{enumerate}
\item $\sig(A) = \prod_{i \in I} \sig(A_i)$,
\item $\states(A) = \prod_{i \in I} \states(A_i)$,
\item $\start(A) = \prod_{i \in I} \start(A_i)$, and
\item $\trans(A)$ contains all $(s,\pi,s^\prime)$ such that for all $i \in I$, if $\pi \in \acts(\sig(A_i))$, then $(s_i,\pi,s_i^\prime) \in \trans(A_i)$ and otherwise, $s_i = s_i^\prime$.
\end{enumerate}
The states of the composite automaton $A$ are vectors of the states of the composed
automata. When an action occurs in
$A$, all of the component automata with that action each take a step simultaneously, as defined
by their individual transition relations. The resulting state differs in each of
the components corresponding to these automata, and the other components are
unchanged. We denote the composition of a small number of
automata using an infix operator, e.g., $A \times B$.

Given the execution $\alpha$ of a composed automaton $A = \prod_{i \in I} A_i$,
we can \textit{project} the execution onto one of the component automata $A_i$.
The execution $\alpha|A_i$ is found by removing all actions from
$\alpha$ that are not actions of $A_i$. The states of $\alpha|A_i$ are
given by the $i$th component of the corresponding state in $\alpha$. The
projection of a trace is defined similarly.

Further, we can write the projection of a state $s$ of $A$ on $A_i$ as $s|A_i$.  Finally, we can also project $\trace(\alpha)$ onto a set of actions $\Pi$ where $\trace(\alpha)|\Pi$ yields the subsequence of $\trace(\alpha)$ containing only actions in $\Pi$.
