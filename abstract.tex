\begin{abstract}
  \Mdl{} is a new consistency model that extends Linearizability
  to allow individual application processes to dispatch concurrent operations. Unlike under Linearizability, concurrent operations in \Mdl{} have a well-defined order based on when they were issued.
  Enabling concurrent operations within a process significantly improves application latency, and guaranteeing ordering between concurrent operations allows more applications to benefit from such an improvement.
  
  Naively using \Mdl{}, however, requires programmers to reason about our new consistency model for their applications.
  To avoid this, we introduce a set of conditions for transforming an application's non-concurrent operations to concurrent ones and then prove a transformed program running on \mdl{} is externally equivalent to the original program running on Linearizability. This enables applications to reap the lower latency of concurrent operations while still reasoning about Linearizability without them.

  No existing system provides \mdl{} across multiple shards, so we design and implement \sys{}.
  Our evaluation compares \sys{} to Multi-Paxos and finds it decreases latency by up to 75\% and up to 84\% in single datacenter and wide-area settings respectively.
\end{abstract}

%   Linearizability is the gold standard consistency model, guaranteeing that operations execute in a total global order across all clients in a manner that respects real-time. Application programmers who build atop linearizable systems experience easy-to-reason-about sequential behavior, almost
%   as if they were programming on a single machine. But even linearizability introduces a trade-off for application
%   programmers: It requires that client processes dispatch only one operation at a time and await its response before issuing the next. This constraint can significantly increase a complex
%   application's end-to-end latency. But clients that do not behave sequentially (e.g., to get lower latency) are not guaranteed a linearizable ordering over their operations, breaking linearizability's easy-to-reason-about semantics.

% %    Linearizability is a widely used and nearly ideal consistency model that provides programmers with a strong abstraction almost as if they were programming on a single machine. Its definition, however, precludes 
  
%   In this paper, we introduce \Multidispatch{} Linearizability (\md{}), a new consistency model that extends Linearizability 
%   to explicitly allow individual clients to issue concurrent operations. Removing this constraint in \md{} improves application latency while guaranteeing a linearization that respects each client's issue order. To demonstrate this, we also design, implement, and evaluate \sys{}, the first multi-shard system to guarantee \md{}. We demonstrate \sys{} can reduce application latency by up to 75\% and \true{90\%} in single and multi-datacenter settings respectively.

