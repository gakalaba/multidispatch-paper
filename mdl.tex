\section{\Multidispatch{} Linearizability}
\label{sec:mdl}

In this section, we define our new consistency model and show how
applications can be transformed to reap the performance benefits it offers while appearing to behave identically to users.
We first describe our model of distributed applications
(\S\ref{sec:mdl:applications} and \S\ref{sec:mdl:systems}) and then define \MDL{} (\S\ref{sec:mdl:def}). We conclude with a description
of how to transform applications and prove our equivalence result
(\S\ref{sec:mdl:equivalence}).

\todo{Make a pass to see if there is anything defined that we don't use and thus that we can cut.}
\subsection{Users, Applications, \& Executions}
\label{sec:mdl:applications}

We leverage the formalism used in RSS~\cite{helt2021rss}. We model a
distributed \textit{application} as a set of $n$ \textit{processes}.
Processes are state machines~\cite{lynch1987ioa,lynch1996da}
that implement application logic by receiving messages from and sending replies
to \textit{users}, performing local computation,
and invoking operations on \textit{systems}.
\wl{Is this somewhat redundant with our definitions in section 2 now?}

An application's processes define a prefix-closed set of \textit{executions},
which are sequences $s_0,\pi_1,s_1,\ldots$ of alternating \textit{states} and
\textit{actions}, starting and ending with a state. An application's state
is an $n$-length vector containing the state of each process.

Each action is a step taken by exactly one process and is one of three types:
\textit{internal}, \textit{input}, or \textit{output}. Internal actions model
local computation. Processes use input and output actions to interact with
users (e.g., responding to an HTTP request) and other
processes (e.g., receiving a remote procedure call).
As we describe in the next section, a subset of the input and output actions
are \textit{system-facing} \textit{invocations} and \textit{responses}, 
respectively.

Users send messages to and receive messages from processes via unidirectional
channels. To send a message to process $P_j$, $P_i$ uses two
actions: first, $P_i$ uses an output action $\sendto_{ij}(m)$
and later, an input action $\sent_{ij}$ occurs, signaling $m$'s 
transmission. Similarly, to receive a message from $P_i$, $P_j$ first
uses an output action $\request_{ij}$ and later, an input action
$\receive_{ij}(m)$ occurs, signaling the receipt of $m$.
(The formalism and proof here ignores inter-process messages for sake of brevity, but the full
proof in Appendix A includes them.)
%Appendix~\ref{sec:equivalence} includes them.)

Such modeling allows our formalism to model scenarios both where human users interact with
directly with the application and where one application is the client
of another, as common in data centers~\cite{veeraraghavan2016kraken,schwarzkopf2018operating}.
In the latter case, however, we assume users do not interact directly with the back-end system (defined below).

Given an execution $\alpha$, we will often refer to an individual process's
\textit{sub-execution}, denoted $\alpha|P_i$. $\alpha|P_i$ comprises
only $P_i$'s actions and the $i$th component of each state in $\alpha$.
In a slight abuse of notation, we use $\alpha|U$ to denote the sub-sequence
of only \textit{user-facing} external actions in $\alpha$. 

\paragraph{Well-formed.} A \textit{well-formed} execution
satisfies the following: (1) Messages are sent before they are received; and
(2) While processes may have multiple outstanding $\request_{ij}$ actions
(at one or more channels) or invocations (at the system), they may not have
both types outstanding simultaneously.
We henceforth only consider well-formed
executions.

\subsection{Systems}
\label{sec:mdl:systems}

Application processes interact with a back-end \textit{system} (e.g., a database or key-value store),
which is defined by its \textit{operations} and a
\textit{specification}~\cite{herlihy1990linearizability,lynch1996da}. An
\textit{operation} comprises pairs of \textit{invocations}, specifying the
operation's arguments, and matching \textit{responses}, containing
return values. The specification is a prefix-closed set of sequences of
invocation-response pairs defining the system's correct behavior in the absence
of concurrency. A sequence $S$ in a specification $\spec$ defines a total order
over its operations, denoted $<_S$.

\subsection{Definition}
\label{sec:mdl:def}

We are now ready to define our new consistency model.
Given an execution $\alpha$, we say an
operation is \textit{complete} if its invocation has a matching response in
$\alpha$. We denote $\complete(\alpha)$ as the maximal subsequence of $\alpha$
comprising only complete operations~\cite{herlihy1990linearizability}.

Given a sequence of complete actions,
$\complete(\alpha)$, we define $\complete(\alpha)||P$ as the
\textit{sequentialization} of $P$'s operations in $\complete(\alpha)$.
$\complete(\alpha)||P$ is found by, for each operation $o$, shifting $o$'s response
$\res(o)$ left in $\complete(\alpha)|P$ until it immediately follows $o$'s
invocation $\inv(o)$.
Importantly, this maintains a process's
\textit{issue order}, the order it invoked its operations.

Two actions in an execution $\alpha$ are
ordered in real time~\cite{herlihy1990linearizability}, denoted
$\pi_1 \rt_\alpha \pi_2$, if $\pi_1$ is a response, $\pi_2$ is an
invocation, and $\pi_1$ precedes $\pi_2$ in $\alpha$.
We can also lift $\rt_\alpha$ to operations in the natural
way.

\paragraph{\Multidispatch{} Linearizability.} An execution $\alpha_1$ satisfies \textit{\multidispatch{} linearizability} if it
can be extended, by adding zero or more responses, to create $\alpha_2$ such that
there exists a sequence $S \in \spec$ where (1) for all $P$,
$S|P = \complete(\alpha_2)||P$, and (2) for all pairs of operations
$o_1$ and $o_2$, $o_1 \rt_{\alpha_1} o_2 \implies o_1 <_S o_2$.

In plain language, an \MDL{} system guarantees it reflects a total order 
over operations. This order is constrained in two ways: First,
it must be consistent with each process's issue order, and second, 
it must be consistent with the real-time order of operations.

\wl{Need a consistency pass to use the same suffix-closed/complete failure \{semantics\} or not everywhere}
%\subsubsection{Suffix-Closed Failures}
\paragraph{Suffix-Closed Failures.}
\label{sec:mdl:def:failures}

\MDL{}'s definition has an important implication for system designers.
In systems where operations can fail (even temporarily before being retried), the
effects of an operation cannot be exposed to other processes until all of its 
predecessors are guaranteed to succeed. Doing so would violate the intuitively
correct behavior of most systems, and formally, this would result in
$S|P \neq \complete(\alpha_2)||P$ for all legal sequential histories
$S \in \spec$.
We refer to this property as \textit{suffix-closed failure semantics}.
\wl{Try to tie this back to the example in the previous section?}
%\true{( example in \cref{subfig:suffix_closed_breaks} from the previous section, the return values r(n)=1, r(m)=0 violate this property because w(n=1) is observed by r(n)=1 and }



In an \MDL{} system, suffix-closed failures must be guaranteed even in the
face of concurrent operations from the same process, which may be to
objects on different shards. As we will see in Section~\ref{sec:mdl:zookeeper},
guaranteeing suffix-closed failures is one way in which existing systems do not implement \mdl{}.  

\wl{Should we punch up this discussion? I think it's a major way we're different from A-Linearizability and its necessary! For instance, we need this for our equivalence result right?! That seems important!}

\subsection{Transforming Applications \& \\External Equivalence}
\label{sec:mdl:equivalence}

In this section, we show how we can transform an application that interacts with
a (\singledispatch{}) linearizable system to take advantage of a comparable
\multidispatch{} system. Importantly, our transformation will ensure that the new
application appears to behave identically to the original application (as far as
the users can tell).

We first define the transformation and several preliminaries. We then
present a condensed form of the proof that the transformation preserves external
equivalence. (A complete proof can be found in Appendix A.) %~\ref{sec:equivalence}.)
For ease of exposition, we focus on the actions within an
execution and assume states can be modified and reordered as needed.

%\subsubsection{Parallelizing Applications}
\paragraph{Parallelizing Applications.}
\label{sec:mdl:transform}

We describe how to transform an application $A$ that is built to
interact with a linearizable system to produce a new application
$A^\prime = \transform(A)$ that can interact with a comparable \MDL{} system.
We attempt to describe this transformation in generic terms, independent of any particular programming language.

The first step is to replace synchronous, system-facing I/O with
futures. This allows operation invocations and responses
to be reordered in $A$ with respect to other actions (i.e., instructions).
The aim is to then move operation invocations earlier such that the application
can issue multiple operations concurrently and reap the performance benefits 
of an \MDL{} system.

\wl{This is the FIRST time the application being concurrent shows up in the paper, this seems too late. Probably needs to be mentioned in the first two sections.}

To produce the new application $A^\prime$, we then move actions in $A$ before prior 
actions provided $A^\prime$ maintains the following:
\textbf{(R1)} the order of data-dependent actions within
each process of $A$,
\textbf{(R2)} the control flow of each process in $A$,
\textbf{(R3)} the issue order of $A$'s operations,
\textbf{(R4)} the order of non-system-facing external
actions in $A$, including messages to/from users,
\textbf{(R5)} the order between a non-system-facing external 
actions and operation invocations (in particular, the first invocation in a
sequence of operations), and
\textbf{(R6)} the order between non-system-facing external actions and operation responses (e.g., by waiting for the successful responses of any outstanding operations before sending a message to a user).

%\subsubsection{Proof of External Equivalence}
\paragraph{Proof of External Equivalence}

Executions $\alpha$ and $\beta$ are
\textit{externally equivalent} if $\alpha|U = \beta|U$. Intuitively,
externally equivalent executions are indistinguishable to users.

\begin{thm}
Let $A$ be an application designed to interact with a linearizable system,
and let $A^\prime = \transform(A)$ be the transformed application. Suppose
$\alpha^\prime$ is an execution of $A^\prime$ that satisfies
\MDL{}. Then there is an externally equivalent execution $\beta$
of $A$ that satisfies linearizability.
\end{thm}

\begin{proof}
The proof proceeds in steps: First, we create an execution $\alpha$ from
$\alpha^\prime$ by essentially undoing the parallelizing transformation in
each process's sub-execution. Second, we produce $\beta$ by fixing the order
of actions across processes to ensure $\beta$ is well-formed and satisfies linearizability.

By the assumption that $A$ is well-formed and the rules of $\transform$,
each $\alpha^\prime | P_i$ comprises sub-sequences of local and
system-facing actions separated by non-system-facing external actions.
Further, by \textbf{R4}, the order of the non-system-facing external
actions in each $\alpha^\prime | P_i$ respects the program order of the
original application $A$. 

We first reorder the actions in each sub-sequence of each $\alpha^\prime | P_i$
to place the local and system-facing actions such that they respect $A$'s program
order. More precisely, for each $\alpha^\prime | P_i$, starting with the right-most
action $\pi$, shift $\pi$ left in $\alpha^\prime$ until it precedes
all other actions $\pi^\prime$ in $\alpha^\prime | P_i$ such that $A$'s program
order requires it. Let $\alpha^\prime$ be the resulting execution.

By the observation about each $\alpha^\prime | P_i$ above, the order of
non-system-facing external actions in $\alpha$ respects $A$'s program order
(because $A^\prime$'s is identical). Further, each $\alpha | P_i$ now
issues its operations sequentially (since $A$'s program order requires it).
Lastly, by \MDL{}'s suffix-closed failure semantics, all of a process's 
completed operations precede any failed ones.

The execution $\alpha$, however, may not satisfy linearizability. We remedy this
and produce the needed $\beta$. To do so, first recall that by assumption the 
original  $\alpha^\prime$ satisfies \MDL{}, so let $<_S$ (for $S \in \spec$) be the total order over the operations in $\alpha^\prime$, as guaranteed by \MDL{}.

Let $\prec_{\alpha}$ be the transitive closure of following pairs of actions
$(\pi_i, \pi_j)$ in $\alpha$:
\begin{enumerate}
    \item $\mathbf{\prec_{P_i}}$\textbf{:} $\pi_i$ precedes $\pi_j$ in some $\alpha | P_i$,
    \item $\mathbf{\prec_U}$\textbf{:} $\pi_i$ precedes $\pi_j$ in $\alpha | U$, or
    \item $\mathbf{\prec_S}$\textbf{:} For each adjacent pair $o_k$, $o_{k+1}$
    in $S$, $\pi_i$ is $o_k$'s response and $\pi_j$ is $o_{k+1}$'s invocation.
\end{enumerate}

We posit $\prec_{\alpha}$ defines an irreflexive partial order
over the actions in $\alpha$. Clearly it is irreflexive; we now prove
it is acyclic and transitive.

For sake of contradiction, assume $\prec_{\alpha}$ contains a cycle, and
let $\pi_1 \prec_{\alpha} \pi_2 \prec_{\alpha} \ldots \prec_{\alpha} \pi_1$
be a shortest cycle that does not include any transitive edges.

We first prove several important properties of the cycle:
\textbf{(P1)} The cycle contains at least two actions.
This follows from the definitions of $\prec_{P_i}$,
$\prec_U$, and $\prec_S$.
\textbf{(P2)} The cycle does not contain two consecutive
$\prec_{P_i}$ edges or two consecutive $\prec_U$ edges. Otherwise, a 
shorter cycle exists since $\prec_{P_i}$ and $\prec_U$ are transitive.
\textbf{(P3)} The cycle does not contain two consecutive $\prec_S$. This follows from the 
definitions of $\prec_S$.

In the remainder of the proof, we leverage the following lemma. 
For sake of brevity, however, we omit its proof.

\begin{lem}
    Let $\pi_1 \prec_{\alpha} \ldots \prec_{\alpha} \pi_n$
    be a sequence of $n$ actions connected any of
    $\prec_{P_i}$, $\prec_U$, and $\prec_S$ edges, and further
    assume $\pi_1$ and $\pi_n$ are send-to or receive-from actions.
    Then $\pi_1$ precedes $\pi_n$ in $\alpha$.
    \label{lem:helper2}
\end{lem}

We continue the proof by cases: First, assume the cycle contains at least one $\prec_U$ edge. Then the cycle can be
written as
$\pi_1 \prec_U \pi_2 \prec_{\alpha} \ldots \prec_{\alpha} \pi_n \prec_{\alpha} \pi_1$. By the definition of $<_U$, clearly $\pi_1$ precedes $\pi_2$ in $\alpha$.

Since users interact with processes using channels, $\pi_2$ and
$\pi_1$ are either receive-from or send-to actions,
so the remainder of the cycle
$\pi_2 \prec_{\alpha} \ldots \prec_{\alpha} \pi_1$
is a sequence of actions connected by $\prec_{P_i}$, $\prec_U$, and $\prec_S$ edges
starting and ending with a send-to or a receive-from action.
Thus, by Lemma~\ref{lem:helper2}, $\pi_2$ precedes $\pi_1$ in $\alpha$,
a contradiction.

Second, assume the cycle does not contain any $\prec_U$ edges.
By Properties \textbf{P2} and \textbf{P3}, the cycle comprises alternating
$\prec_{P_i}$ and $\prec_S$ edges. Also, all $P_i$ in the $\prec_{P_i}$
must be distinct, otherwise a shorter cycle must exist.
Lastly, since they alternate with $\prec_S$ edges,
for each $\pi_i \prec_{P_i} \pi_{i+1}$, we know 
$\pi_i$ is an invocation and $\pi_{i+1}$ is a response.

By definition of $\prec_{P_i}$, $\pi_i$ precedes $\pi_{i+1}$ in $\alpha | P_i$.
Since each process’s system interactions occur sequentially in $\alpha$,
$\pi_{i}$ must also precede $\pi_{i+1}$'s corresponding invocation.
Since the transformation between $A$ and $A^\prime$ preserves each process’s
invocation order (as does MDL), this implies $\op(\pi_{i})$ is ordered in $S$
before $\op(\pi_{i+1})$.

As a result, for each $\pi_i \prec_{P_i} \pi_{i+1}$ in our cycle, we can replace it
with a sequence of $\prec_{P_i}$ and $\prec_S$ edges that includes the invocations 
and responses of all operations in $S$ between $\op(\pi_i)$ and $\op(\pi_{i+1})$
(if any). This (no longer shortest) cycle implies there is a cycle over operations 
in $S$. But this contradicts the definition of $S$, which is a 
total order. Thus, $\prec_{\alpha}$ is acyclic. Combined with the
definitions of $\prec_{P_i}$, $\prec_U$, and $\prec_S$, this also
shows it is irreflexive.

Let $\beta$ be a topological sort of $\prec_{\alpha}$ on the actions in
$\alpha$. To conclude the proof, we show $\beta$ is well-formed,
satisfies linearizability, and is externally equivalent to $\alpha^\prime$.

The fact that $\beta$ is well-formed follows from the definition of
$\prec_{\alpha}$. It preserves the order of user-facing actions,
which were unaltered in the transformation from $\alpha^\prime$ to $\alpha$.
Similarly, it preserves the order of each message's send-to and receive-from actions,
and their order was also unaltered in producing $\alpha$. Finally, the original
transformation to $\alpha$ placed the actions of each $\alpha | P_i$ in
the order dictated by $A$'s program order, and this was preserved by
$\prec_{\alpha}$ in producing $\beta$.

By the initial reordering to produce $\alpha$ and the definition of
$\prec_{\alpha}$, the operations across all processes in $\beta$
are sequential in the order defined by $S$. Since $<_S$ defines a
total order over all operations, $\beta$ thus satisfies linearizability.

Finally, since the initial transformation from $\alpha^\prime$ to $\alpha$
did not alter the order of any user-facing actions, clearly
$\alpha^\prime | U = \alpha | U$. Then by the definition of $\prec_{\alpha}$, 
$\alpha | U = \beta | U$. $\beta$ is thus externally equivalent to $\alpha$.
\end{proof}
