\section{Users, Applications, \&Executions}
\label{sec:mdl:applications}

We leverage the formalism used in RSS~\cite{helt2021rss}. We model a
distributed \textit{application} as a set of $n$ \textit{processes}.
Processes are state machines~\cite{lynch1987ioa,lynch1996da}
that implement application logic by receiving messages from and sending replies
to \textit{users}, performing local computation,
and invoking operations on \textit{systems}.
\wl{Is this somewhat redundant with our definitions in section 2 now?}

An application's processes define a prefix-closed set of \textit{executions},
which are sequences $s_0,\pi_1,s_1,\ldots$ of alternating \textit{states} and
\textit{actions}, starting and ending with a state. An application's state
is an $n$-length vector containing the state of each process.
%
Each action is a step taken by exactly one process and is one of three types:
\textit{internal}, \textit{input}, or \textit{output}. Internal actions model
local computation. Processes use input and output actions to interact with
users (e.g., responding to an HTTP request) and other
processes (e.g., receiving a remote procedure call).
As we describe in the next section, a subset of the input and output actions
are \textit{system-facing} \textit{invocations} and \textit{responses}, 
respectively.

Users send messages to and receive messages from processes via unidirectional
channels. To send a message to process $P_j$, $P_i$ uses two
actions: first, $P_i$ uses an output action $\sendto_{ij}(m)$
and later, an input action $\sent_{ij}$ occurs, signaling $m$'s 
transmission. Similarly, to receive a message from $P_i$, $P_j$ first
uses an output action $\request_{ij}$ and later, an input action
$\receive_{ij}(m)$ occurs, signaling the receipt of $m$.
(The formalism and proof here ignores inter-process messages for sake of brevity, but the full
proof in Appendix A includes them.)
%Appendix~\ref{sec:equivalence} includes them.)

Such modeling allows our formalism to model scenarios both where human users interact with
directly with the application and where one application is the client
of another, as common in data centers~\cite{veeraraghavan2016kraken,schwarzkopf2018operating}.
In the latter case, however, we assume users do not interact directly with the back-end system (defined below).

Given an execution $\alpha$, we will often refer to an individual process's
\textit{sub-execution}, denoted $\alpha|P_i$. $\alpha|P_i$ comprises
only $P_i$'s actions and the $i$th component of each state in $\alpha$.
In a slight abuse of notation, we use $\alpha|U$ to denote the sub-sequence
of only \textit{user-facing} external actions in $\alpha$. 

\paragraph{Well-formed.} A \textit{well-formed} execution
satisfies the following: (1) Messages are sent before they are received; and
(2) While processes may have multiple outstanding $\request_{ij}$ actions
(at one or more channels) or invocations (at the system), they may not have
both types outstanding simultaneously.
We henceforth only consider well-formed
executions.

