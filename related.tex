\section{Related Work}
\label{sec:related}

\paragraph{Consistency Models.}
The core difference between Linearizability and \mdl{} is that \mdl{} allows multi-dispatch and enforces issue-ordering for operations from the same application process.
Using issue ordering as the constraint is intuitive and thus conceptually matches earlier work.
Session guarantees~\cite{terry1994session} introduces four guarantees for operations within a given session: read-your-writes, monotonic reads, writes-follow-reads, and monotonic writes. When a session corresponds to an application process handling a user request the combination of these four guarantees ensures issue ordering.
Similarly, A-Linearizability~\cite{hunt2010zookeeper} requires an issue-order for multi-dispatched writes.

This earlier work, however, does not provide `strong' consistency comparable to Linearizability.
Session guarantees do not ensure any real-time order, e.g., a write that finished yesterday might not been seen by a read issued today.
A-Linearizability, despite its name, only provides Linearizability-like consistency for write operations and also has stale reads.
Thus, \Mdl{} is distinct from this earlier work it builds on by providing `strong' Linearizability-like consistency for all operations.

Sequential consistency was originally described in the context of
multi-processors~\cite{lamport1979sequential}. It requires that
the system reflects a total order over all operations that is 
consistent with each processor's `program order.' It is often
interpreted as allowing processors to issue multiple memory 
operations concurrently, as is common in computer architecture.
But in the distributed systems community, it has often
included the assumption that each process issues operations 
sequentially~\cite{attiya1993seqlin}.

But like other, weaker consistency
models~\cite{ahamad1995causal,lloyd2011cops,terry1995bayou,deCandia2007dynamo}, 
sequential consistency introduces a trade-off. Although it enables
better-performing systems, it makes building correct applictions more difficult. In contrast, through its transformation
and accompanying equivalence result, \MDL{} offers transformed
applications better performance while guaranteeing they behave 
identically to the original application.

\paragraph{Equivalence Results.}
Other works have used the idea of equivalence to various
ends~\cite{goldman1993unifiedModel,lundelius1984clocksync,
fischer1985flp,attiya1993seqlin},
such as proving bounds on clock
synchronization~\cite{lundelius1984clocksync}.
Most similar to our work is Helt et al.~\cite{helt2021rss}, which shows that their consistency
models provided identical correctness guarantees for applications
as strict serializability~\cite{papadimitriou1979serializability} and linearizability~\cite{herlihy1990linearizability}.
They refer to this as ``invariant equivalence.'' 

External equivalence provides a proves a different property.
It does not guarantee that the original and transformed applications
will transition through the same states and thus obey the same
application invariants. Instead, it guarantees that users cannot distinguish between the two
application versions.
Conversely, invariant-equivalent applications may exhibit different
user-facing behaviors (referred to as ``anomalies'' in Helt et
al.~\cite{helt2021rss}), so they are not necessarily externally equivalent.
We leave further investigation into
\MDL{}'s impacts on application invariants to future work.

\paragraph{Linearizable Systems.} Many protocols have been developed to provide fault-tolerant, linearizable storage.
State-of-the-art leader-based
protocols~\cite{ongaro2014raft,lamport1998paxos,oki1988vr},
suffer in wide-area deployments due to the extra
latency incurred between application processes and (potentially far)
leaders.
There is a body of work that improves the performance of
linearizable systems in the wide
area~\cite{mao2008mencius,moraru2013epaxos,burke2020gryff} and that can sometimes avoid the latency penalty from a far-away leader.
As a leader-based protocol \sys{} must pay the penalty for far-away leaders.
An interesting avenue for future work is removing this penalty by adding \mdl{} to some of these designs.

\paragraph{Transactional Systems.}
Transactional consistency models, like strict serializability,
allow application processes to group their operations into
transactions~\cite{papadimitriou1979serializability}. Such
systems generally guarantee that the operations in a transaction
will take effect atomically, as one indivisible unit (although weaker 
guarantees also exist~\cite{adya1999weakcons}).
Compared to \MDL{}'s issue order guarantee, transactional atomicity 
is stronger since it prevents other processes from observing 
intermediate states where some of the operations in a transaction 
have taken effect but not others. 

% A closer investigation
% of the relationship between these guarantees (and its
% implications for application correctness) is left to future work.

Some transactional consistency
models~\cite{papadimitriou1979serializability,adya1999weakcons} similarly assume 
each application process issues one transaction at a time. This may 
make it possible to combine the ideas here with that of transactions. 
This may offer similar improvements in end-to-end 
performance for application built atop existing transactional 
systems~\cite{thomson2014calvin,mahmoud2013replicatedCommit,zhang2018tapir,mu2014rococo,mu2016janus,kraska2013mdcc,ren2019slog,taft2020crdb,yan2018carousel}.





