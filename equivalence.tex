\section{External Equivalence Proof}
\label{sec:equivalence}

In this section, we first define some preliminaries and then use them to show
that an application built atop a linearizable service can be transformed to take
advantage of the concurrency offered by \multidispatch{} linearizability such that
the two versions of the application are externally equivalent. Our formalism
closely aligns with that of prior work~\cite{helt2021rss}, which leverages the
formalism of I/O automata~\cite{lynch1987ioa,lynch1996da}.

\subsection{Preliminaries}
\label{sec:equivalence:preliminaries} 

\subsubsection{I/O Automata}
\label{sec:equivalence:preliminaries:ioa}

We model each component in our system as an I/O automaton
(IOA)~\cite{lynch1987ioa,lynch1996da}, a type of state machine. Each
transition of an IOA is labeled with an \textit{action}, which can be an
\textit{input}, \textit{output}, or \textit{internal} action. Input and output
actions allow the automaton to interact with other IOA and the environment. We
assume input actions are not controlled by an IOA---they may arrive at any time.
Conversely, output and internal actions are \textit{locally controlled}---an IOA defines
when they can be performed.

To specify an IOA, we must first specify its \textit{signature}. A signature $S$ is
a tuple comprising three disjoint sets of actions: input actions $\actin(S)$,
output actions $\actout(S)$, and internal actions $\actint(S)$. We also define
$\localacts(S) = \actout(S) \cup \actint(S)$ as the set of locally controlled actions, $\extacts(S) = \actin(S) \cup \actout(S)$ as the set of external actions, and $\acts(S) = \actin(S) \cup \actout(S) \cup \actint(S)$ as the set of all actions.

Formally, an I/O automaton $A$ comprises four items:
\begin{enumerate}
\item a signature $\sig(A)$,
\item a (possibly infinite) set of states $\states(A)$,
\item a set of start states $\start(A) \subseteq \states(A)$, and
\item a transition relation $\trans(A) \subseteq \states(A) \times
  \acts(\sig(A)) \times \states(A)$.
\end{enumerate}
Since inputs may arrive at any time, we assume that for every state $s$ and
input action $\pi$, there is some $(s, \pi, s^\prime) \in \trans(A)$.

An \textit{execution} of an I/O automaton $A$ is a finite or infinite
sequence of alternating states and actions $s_0,\pi_1,s_1,\ldots$ such that for
each $i \geq 0$, $(s_i, \pi_{i+1}, s_{i+1}) \in \trans(A)$ and $s_0 \in \start(A)$. Finite executions always end with a state.

Given an execution $\alpha$, we can also define
its \textit{schedule} $\sched(\alpha)$, which is the subsequence of just
the actions in $\alpha$. Similarly, an execution's \textit{trace}
$\trace(\alpha)$ is the subsequence of just the external actions $\pi \in
\extacts(A)$.

\subsubsection{Composition and Projection}
\label{sec:equivalence:preliminaries:composition}

To compose
two IOA, they must be \textit{compatible}. Formally, a finite set of
signatures $\{S_i\}_{i \in I}$ is compatible if for all $i,j \in I$ such that $i \neq j$:
\begin{enumerate}
\item $\actint(S_i) \cap \acts(S_j) = \emptyset$, and
\item $\actout(S_i) \cap \actout(S_j) = \emptyset$.
\end{enumerate}
A finite set of automata are compatible if their signatures are compatible.

Given a set of compatible signatures, we define their \textit{composition} $S =
\prod_{i \in I} S_i$ as the signature with $\actin(S) = \bigcup_{i \in I}
\actin(S_i) - \bigcup_{i \in I} \actout(S_i)$, $\actout(S) = \bigcup_{i \in I}
\actout(S_i)$, and $\actint(S) = \bigcup_{i \in I} \actint(S_i)$.

The composition of a set of compatible IOA yields the automaton $A =
\prod_{i \in I} A_i$ defined as follows:
\begin{enumerate}
\item $\sig(A) = \prod_{i \in I} \sig(A_i)$,
\item $\states(A) = \prod_{i \in I} \states(A_i)$,
\item $\start(A) = \prod_{i \in I} \start(A_i)$, and
\item $\trans(A)$ contains all $(s,\pi,s^\prime)$ such that for all $i \in I$, if $\pi \in \acts(\sig(A_i))$, then $(s_i,\pi,s_i^\prime) \in \trans(A_i)$ and otherwise, $s_i = s_i^\prime$.
\end{enumerate}
The states of the composite automaton $A$ are vectors of the states of the composed
automata. When an action occurs in
$A$, all of the component automata with that action each take a step simultaneously, as defined
by their individual transition relations. The resulting state differs in each of
the components corresponding to these automata, and the other components are
unchanged. We denote the composition of a small number of
automata using an infix operator, e.g., $A \times B$.

Given the execution $\alpha$ of a composed automaton $A = \prod_{i \in I} A_i$,
we can \textit{project} the execution onto one of the component automata $A_i$.
The execution $\alpha|A_i$ is found by removing all actions from
$\alpha$ that are not actions of $A_i$. The states of $\alpha|A_i$ are
given by the $i$th component of the corresponding state in $\alpha$. The
projection of a trace is defined similarly.

Further, we can write the projection of a state $s$ of $A$ on $A_i$ as $s|A_i$.  Finally, we can also project $\trace(\alpha)$ onto a set of actions $\Pi$ where $\trace(\alpha)|\Pi$ yields the subsequence of $\trace(\alpha)$ containing only actions in $\Pi$.

\subsubsection{Channels}
\label{sec:equivalence:preliminaries:channels}

\begin{table}[!t]
  \centering
  \begin{tabular}{l l l}
    \multicolumn{3}{l}{\textbf{Signature:}} \\ \hline
    Inputs: & & Outputs: \\
    $\sendto(m)_{ij}, m \in M$ & & $\sent_{ij}$ \\
    $\recvfrom_{ij}$ & & $\receive(m)_{ij}, m \in M$ \\ \hline
    \\
    \multicolumn{3}{l}{\textbf{States:}} \\ \hline
    \multicolumn{3}{l}{$Q$, a FIFO queue of elements of $M$, initially empty} \\
    \multicolumn{3}{l}{$e$, a Boolean, initially \textit{false}} \\
    \multicolumn{3}{l}{$r$, a Boolean, initially \textit{false}} \\ \hline
    \\
    \multicolumn{3}{l}{\textbf{Actions:}} \\ \hline
    \multicolumn{2}{l|}{$\sendto_{ij}(m)$:} & $\sent_{ij}$: \\
    \multicolumn{2}{l|}{Precondition: \textit{true}} & Precondition: $e$ \\
    \multicolumn{2}{l|}{Effect: $\textsc{Push}(Q,m); e \gets \textit{true}$} & Effect: $e \gets \textit{false}$ \\
    \hline \hline
    $\recvfrom_{ij}$: & \multicolumn{2}{|l}{$\receive_{ij}(m)$:} \\
    Precondition: \textit{true} & \multicolumn{2}{|l}{Precondition: $r \land m = \text{Head}(Q)$} \\
    Effect: $r \gets \textit{true}$ & \multicolumn{2}{|l}{Effect: $\textsc{Pop}(Q); r \gets \textit{false}$} \\ \hline
  \end{tabular}
  \captionof{figure}{Buffered Channel I/O Automaton}
  \label{fig:buffered-channel-ioa}
\end{table}

Each pair of processes in our system communicates via a pair of FIFO
channels that are asynchronous, reliable, and buffered. Each
channel's signature, states, and actions are specified in
Figure~\ref{fig:buffered-channel-ioa}.

We denote the channel that process $i$ uses to send messages to process $j$ as
$C_{ij}$. $C_{ij}$ has two sets of input actions, $\sendto_{ij}(m)$ and
$\recvfrom_{ij}$, and two sets of output actions, $\sent_{ij}$ and
$\receive_{ij}(m)$, for all $m$ in some space of messages $M$. Process $i$ has
corresponding output actions, $\sendto_{ij}(m)$ and $\recvfrom_{ji}$, and input
actions, $\sent_{ij}$ and $\receive_{ji}(m)$, for all other processes $j$ and
messages $m$. To send a message to process $j$, process $i$ takes a
$\sendto_{ij}$ step, and $C_{ij}$ subsequently takes a $\sent_{ij}$ step.
Similarly, to receive a message from $C_{ij}$, process $j$ takes a
$\recvfrom_{ij}$ step, and $C_{ij}$ subsequently takes a $\receive_{ij}(m)$ step.

The modeling of buffering in the channels differs from past
work~\cite{lynch1996da}. There, receive-from actions are omitted, and received
actions are modeled as output actions of channels and corresponding input
actions of processes. This implies processes cannot control when they
change their state in response to a message.

But in real applications, this is unrealistic. Although the network stack of a
machine may accept and process a packet at any time, application code controls
when it processes the contained message. For instance, the packet's contents remain in a kernel buffer until the application performs a read system call
on a network socket. This control is essential to our proof as it ensures
application processes do not receive messages while waiting for responses from
services.

We say an execution $\alpha$ of a channel $C_{ij}$ is \textit{well-formed} if
(1) $\trace(\alpha) | \{\sendto_{ij}(m)\}_{m \in M} \cup \{\sent_{ij}\}$ is a
sequence of alternating send-to and sent actions, starting with a send-to; and
similarly (2) $\trace(\alpha) | \{\recvfrom_{ij}\} \cup \{\receive_{ij}(m)\}_{m
  \in M}$ is a sequence of alternating receive-from and received actions,
starting with a receive-from.

\subsubsection{Types and Services}
\label{sec:equivalence:preliminaries:services}

Processes in our system interact with \textit{services}, each with a specified
\textit{type}~\cite{herlihy1990linearizability,lynch1996da}. A service's type
$\type$ defines its set of possible \textit{values} $\vals(\type)$, an
\textit{initial value} $v_0 \in \vals(\type)$, and the \textit{operations}
$\ops(\type)$ that can be invoked on the service. Each operation $o \in
\ops(\type)$ is defined by a pair of sets of actions: \textit{invocations} $\invs(o)$ and
\textit{responses} $\resps(o)$. Each contains subscripts denoting a unique service name and a process index. An invocation and response \textit{match} if their subscripts are equal.
Finally, each service has a \textit{sequential specification} $\spec$, a
prefix-closed set of sequences of matching invocation-response pairs.

For example, consider a read/write register $x$ that supports a set of $n$
processes and whose values is the set of integers. The read operation would then
be defined with invocations $\{\textsc{read}_{i,x}\}$ and responses
$\{\textsc{ret}_{i,x}(j)\}$ for all $0 \leq i \leq n$ and $j \in \mathcal{N}$.
Similarly, the write operation would have invocations
$\{\textsc{write}_{i,x}(j)\}$ and responses $\{\textsc{ack}_{i,x}\}$. Finally,
its sequential specification would be the set of all sequences of reads and
writes such that reads return the value written by the most recent write or the
initial value if none exists.

We can also compose types $\{\type_x\}_{x \in X}$ and sequential
specifications $\{\spec_x\}_{x \in X}$. Formally, the values of $\type = \prod_{x
  \in X} \type_x$ are vectors of the values of the composed types. $\ops(\type)$
is the union of those of the composed types. Finally, the composite sequential
specification $\spec = \prod_{x \in X} \spec_x$ is the set of all interleavings of the the invocation-response pairs of the component
specifications $\spec_x$.

\subsubsection{System Model}
\label{sec:equivalence:preliminaries:model}

We model a distributed application as the composition of two finite sets of I/O
automata: processes and channels. $n$ denotes the number of processes, so there are $n^2$ channels. The processes execute the application's code by performing local computation, exchanging messages via channels, and performing invocations on and receiving responses from services.

In the results below, we are interested in reasoning about which process invariants hold while assuming various correctness conditions of the services they interact with. Thus, we do not model services as IOA. Instead, we assume the processes interact with a (possibly composite) service with an arbitrary type $\type$ and $\spec$ defined for $n$ processes.

For each operation $o \in \ops(\type)$, process $P_i$ is then assumed to have an output action for every invocation action in $\invs(o)$ with process index $i$. Similarly, $P_i$ has an input action for every response action in $\resps(o)$ with process index $i$. We refer to these input and output actions as a process's \textit{system-facing} actions $\system(P_i)$.

To model stop failures, we assume each process $P_i$ has an input
action $\textit{stop}_i$ such that after receiving it, $P_i$ ceases taking
steps. If $\textit{stop}_i$ occurs while $P_i$ is waiting for a response from a service, then we assume the service does not return a response. But the operation may still cause a service's state to change, and this change may be visible to operations by other processes.

Finally, to allow the distributed application to receive input from and return values to its environment (e.g., users), we assume each $P_i$ has a set of \textit{user-facing} actions $\user(P_i)$. Similar to a process's interactions with services, a user's interaction with a process is modeled through input-output pairs of user-facing actions.

We make two final assumptions about the processes: First, we assume that while
each process has access to a local clock, which is part of its state, and may
set local timers, which are internal actions, the process makes no assumptions
about the drift or skew of its clock relative to others. Second, processes only
invoke an operation on a service when they have no outstanding send-to or
receive-from actions at any channels.

Let $P = \prod_{i \in I} P_i$ be the composition of the $n$ processes and $C = \prod_{1 \leq i \leq n} \prod_{1 \leq j \leq n} C_{ij}$ be the composition of $n^2$ channels. Let $\alpha$ be an execution of the distributed application $P \times C$. $\alpha$ is \textit{well-formed} if it satisfies three criteria:
First, for all $P_i$, $\trace(\alpha) | \system(P_i)$ must be a sequence of alternating invocation and matching response actions, starting with an invocation. Second, for all $C_{ij}$, $\alpha | C_{ij}$ is well-formed. Third, for all $P_i$, $P_i$ does not take an output step while waiting for a response from some service. 

\subsubsection{External Equivalence}
\label{sec:equivalence:preliminaries:equivalence}

\subsubsection{Transformation}
\label{sec:equivalence:preliminaries:transform}

\subsection{Proof}
\label{sec:equivalence:proof}

\begin{theorem}
  Suppose $A$ is an IOA defining an application that interacts with a linearizable service,
  and let $A^\prime = \transform(A)$ be the IOA defining the transformed application that
  interacts with a comparable \multidispatch{} linearizable service. Suppose $\alpha$ is a
  well-formed execution of $U \times C \times A^\prime$ that satisfies \multidispatch{} real-time
  precedence, then there exists a well-formed execution $\beta$ of $U \times C \times A$ that
  satisfies \singledispatch{} real-time precedence such that $\alpha | U = \beta | U$.
\end{theorem}

\begin{proof}
  Let $U$, $A$, and $A^\prime$ be defined as above, and let $\alpha$ be an arbitrary,
  well-formed execution of $U \times C \times A^\prime$ that satisfies \multidispatch{}
  real-time precedence. We proceed by constructing $\beta$.

  By assumption, $\alpha$ satisfies \multidispatch{} real-time precedence, so let $S$
  be a total order over the operations in $\alpha | A^\prime$, as guaranteed by \MDL{}.

  By the original restrictions on $A$ and rules 4, 5, 6, and 7 in the definition of $\transform$,
  each $\alpha | P_i$ comprises sub-sequences of local and system-facing actions separated by
  non-system-facing external actions. Further, by rule 5 of $\transform$, the order of the
  non-system-facing external actions in each $\alpha | P_i$ respects the program order of the
  original application $A$. 

  We first reorder the actions in each sub-sequence of each $\alpha | P_i$ by reversing the
  transformation, placing the local and system-facing actions such that they respect the program
  order of $A$. More precisely, for any two actions $\pi_1$, $\pi_2$ that were reordered by $\transform$
  such that $\pi_1$ precedes $\pi_2$ in $A$'s program order but $\pi_2$ precedes $\pi_1$ in $A^\prime$'s,
  shift $\pi_1$ left in $\alpha$ until it precedes $\pi_2$. Let $\alpha^\prime$ be the resulting execution.

  By the observation about each $\alpha | P_i$ above, the order of non-system-facing external actions
  in $\alpha^\prime$ respects $A$'s program order (because $A^\prime$'s is identical). Further,
  each $\alpha^\prime | P_i$ now issues system operations sequentially. These operations, however,
  may not be linearizable.

  To produce a well-formed execution $\beta$, we next let $\caused_{\alpha^\prime}$ be the set of the
  following pairs $(\pi_i, \pi_j)$ of actions in $\alpha^\prime$:
  \begin{enumerate}
    \item $\mathbf{<_M}$\textbf{:} $\pi_i$ is a send action for some message $M$ (between two useres,
    from a user to a process, a process to a user, or between two processes) and $\pi_j$
    is its corresponding receive action,
    \item $\mathbf{<_{P_i}}$\textbf{:} $\pi_i$ precedes $\pi_j$ in some $\alpha^\prime | P_i$,
    \item $\mathbf{<_U}$\textbf{:} $\pi_i$ precedes $\pi_j$ in $\alpha^\prime | U$,
    \item $\mathbf{<_S}$\textbf{:} For each adjacent pair of operations $o_k$, $o_{k+1}$ in $S$, $\pi_i$ is
    the response action of $o_k$ and $\pi_j$ is the invocation action of $o_{k+1}$, or
    \item \textbf{Transitivity:} there exists some action $\pi_k$ such that $(\pi_i, \pi_k)$
    and $(\pi_k, \pi_j)$ are in $\caused_{\alpha^\prime}$.
  \end{enumerate}

  We posit $\caused_{\alpha^\prime}$ defines an irreflexive partial order over the actions
  in $\alpha^\prime$. Clearly it is irreflexive and transitive; we now prove it is acyclic.

  For sake of contradiction, assume $\caused_{\alpha^\prime}$ contains at least one cycle. Let
  $\pi_1 \caused_{\alpha^\prime} \pi_2 \caused_{\alpha^\prime} \ldots \caused_{\alpha^\prime} \pi_1$
  be one of the shortest such cycles that does not include any transitive edges.

  Before proving our contradiction, we first prove several important properties of the cycle,
  as well as two lemmas.

  \noindentparagraph{Property 1.} The cycle contains at least two actions.
  This follows from the definitions of $<_M$, $<_{P_i}$, $<_U$, and $<_S$.

  \noindentparagraph{Property 2.} The cycle does not contain two consecutive
  $<_{P_i}$ edges or two consecutive $<_U$ edges. Otherwise, a shorted cycle
  exists since $<_{P_i}$ and $<_U$ are transitive.

  \noindentparagraph{Property 3.} The cycle does not contain two consecutive
  $<_M$ edges or two consecutive $<_S$. This follows from the definitions of
  $<_M$ and $<_S$.

  \begin{lemma}
    Let $\pi_1 \caused_{\alpha^\prime} \ldots \caused_{\alpha^\prime} \pi_n$
    be a sequence of $n$ actions connected only by
    $<_{P_i}$, $<_U$, and $<_S$ edges (i.e., no $<_M$ edges), and further
    assume $\pi_1$ is a message receive action and $\pi_n$ is a message
    send action. Then, $\pi_1 \rt_{\alpha^\prime} \pi_n$.
    \label{lemma:helper1}
  \end{lemma}

  \begin{proof}
    TODO
  \end{proof}

  \begin{lemma}
    Let $\pi_1 \caused_{\alpha^\prime} \ldots \caused_{\alpha^\prime} \pi_n$
    be a sequence of $n$ actions connected any of $<_M$, $<_{P_i}$, $<_U$,
    and $<_S$ edges, and further assume $\pi_1$ is a message receive action
    and $\pi_n$ is a message send action. Then, $\pi_1 \rt_{\alpha^\prime} \pi_n$.
    \label{lemma:helper2}
  \end{lemma}
    TODO
  \begin{proof}
    
  \end{proof}

  We now leverage these lemmas to conclude the proof, so recall that
  $\pi_1 \caused_{\alpha^\prime} \pi_2 \caused_{\alpha^\prime} \ldots \caused_{\alpha^\prime} \pi_1$
  is a shortest cycle. We proceed by cases:

  First, assume the cycle contains at least one $<_M$ edge. Then the cycle can be written as
  $\pi_1 <_M1 \pi_2 \caused_{\alpha^\prime} \ldots \pi_2 \caused_{\alpha^\prime} \pi_n \pi_2 \caused_{\alpha^\prime} \pi_1$ 
  by re-indexing the actions as needed. By the assumption that $\alpha$ is well-formed,
  the fact that $\alpha^\prime$ preserved the order of $\alpha$'s non-system-facing actions,
  and the definition of $\pi_1 <_M \pi_2$, it is clear that $\pi_1 \rt_{\alpha^\prime} \pi_2$.
  But $\pi_2 \caused_{\alpha^\prime} \ldots \pi_2 \caused_{\alpha^\prime} \pi_n \pi_2 \caused_{\alpha^\prime} \pi_1$
  is a sequence of actions connected by $<_M$, $<_{P_i}$, $<_U$, and $<_S$ edges
  starting with a receive action ($\pi_2$) and ending with a send action ($\pi_1$),
  so by Lemma~\ref{lemma:helper2}, $\pi_2 \rt_{\alpha^\prime} \pi_1$, which is a contradiction.

  Now assume the cycle does not contain any $<_M$ edges. Since there are no messages in
  the cycle, there are two sub-cases: the cycle comprises either only $<_U$ edges or alternating
  $<_{P_i}$ and $<_S$ edges (by Properties 2 and 3).

  In the former case, a cycle contradicts the definition of $<_U$, so we turn to the latter.
  Note that all $P_i$ in the $<_{P_i}$ must be distinct, otherwise a shorter cycle must exist.
  For each $\pi_i <_{P_i} \pi_{i+1}$, we know that $\pi_i$ is an invocation action and
  $\pi_{i+1}$ is a response action. Further, by similar reasoning as in Lemma~\ref{lemma:helper1},
  we know that $\op(\pi_i)$ must precede $\op(\pi_{i+1})$ in $S$. Thus, for each $\pi_i <_{P_i} \pi_{i+1}$,
  we can expand it to include the invocations and responses of all operations in $S$ between $\op(\pi_i)$
  and $\op(\pi_{i+1})$ (if any). This (no longer shortest) cycle implies there is a cycle
  over operations in $S$. But this contradicts the definition of \MDL{}, which guarantees a total order
  over all operations and which was used to instantiate $S$. Thus, $\caused_{\alpha^\prime}$ is acyclic,
  and combined with the definitions of $<_M$, $<_{P_i}$, $<_U$, and $<_S$, this also shows
  it is irreflexive.

  Let $\beta$ be a topological sort of $\caused_{\alpha^\prime}$ on the actions in
  $\alpha^\prime$. To conclude the proof, we show $\beta$ is well-formed,
  satisfies \singledispatch{} real-time precedence, and is externally equivalent to $\alpha$.

  The fact that $\beta$ is a well-formed execution of $U \times C \times A$ follows
  from the definition of $\caused_{\alpha^\prime}$. It preserves the order of actions
  in $U$, which were unaltered in the transformation from $\alpha$ to $\alpha^\prime$.
  Similarly, it preserves the order of each message's send-to and receive-from actions,
  and their order was also unaltered in producing $\alpha^\prime$. Finally, the original
  transformation to $\alpha^\prime$ placed the actions of eacch $\alpha^\prime | P_i$ in
  the order dictated by $A$'s program order, and this was preserved by
  $\caused_{\alpha^\prime}$ in producing $\beta$.

  By the initial reordering of to produce $\alpha^\prime$ and the definition of
  $\caused_{\alpha^\prime}$, the system-facing actions are now \singledispatch{} in $\beta$.
  In fact, by the defintion $\caused_{\alpha^\prime}$, accesses are now sequential
  (across all processes) in the order defined by $S$. Since $S$ defined a total order over all
  operations, $\beta$ thus satisfies \singledispatch{} real-time precedence.

  Finally, since the initial transformation from $\alpha$ to $\alpha^\prime$ did not alter
  the order of actions in $U$, clearly $\alpha | U = \alpha^\prime | U$. Then by the defintion
  of $\caused_{\alpha^\prime}$, $\alpha^\prime | U = \beta | U$. Thus, $\beta$ is externally
  equivalent to $\alpha$.
\end{proof}