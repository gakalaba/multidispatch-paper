\section{Implementation}
\label{sec:implementation}
We implemented \system by modifying the existing implementation of basic \mpaxos from the EPaxos codebase ~\ref{} written in Go. To provide a fair comparison against the \mpaxos baseline, we change as little as possible within the existing framework. Both \system and Paxos have single threaded leaders and replicas, neither have operation type-specific optimizations (such as read leases), both use the thrifty optimization, and neither use batching for fault-tolerance. The existing framework implements a replicated key-value store.

We extend the message types each replica handles to include our coordination and final ordering messages, and we use a Golang ticker object ~\cite{} to implement epochs. It's worth noting the inaccuracy of Golang tickers for values below 2ms ~\ref{}, thus for most of our experiments (which have an epoch shorter than 2ms) there is noticeable variability for epoch length.

Since \sdl is composable, the unmodified \mpaxos clients submit requests to the leader of a single shard. We extended the client library so that clients issue requests to multiple shard leaders, since we evaluate the multi-sharded setting. Clients keep track of a per-shard sequence number, and when requests are concurrent, they submit coordination requests to the shard leaders holding predecessor requests.

The code and experimental scripts are available online ~\cite{}.
